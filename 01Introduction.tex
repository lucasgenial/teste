\section{Introdução}

\begin{frame}
	\frametitle{Introdução}
	\begin{itemize}
		\item A abordagem para estimação de estados em sistemas dinâmicos desenvolvida por Rudolf Kalman em \cite{Kalman1960} tem sido utilizada em diversos estudos na literatura.
		\item No decorrer da sua utilização algumas limitações foram apresentadas, pode-se exemplificar:
		\begin{itemize}
			\item problemas de convergência causados devido à falta de precisão dos algoritmos numéricos ou modelagem não apropriada dos sistemas a serem estimados \cite{Jesus2007};
			\item problema numérico na filtragem relacionado a condições iniciais desconhecidas do filtro Kalman.
		\end{itemize}
	\end{itemize}
\end{frame}

\begin{frame}
	\frametitle{Introdução}
	\begin{itemize}
		\item Na busca por soluções foram desenvolvidos novos algoritmos para diferentes implementações do filtro de Kalman, entre eles destacam-se a utilização de algoritmo \textit{array}, introduzido por Potter \cite{POTTER1963}, e o filtro na forma da informação \cite{Anderson1979}.
	\end{itemize}
\end{frame}